\documentclass[12pt]{article}
\usepackage{graphicx}

\begin{document}

\title{\textbf{Summer Work Report}}
\author{Dmitri Mokhov \\
        \small{(mokhovd@fnal.gov, mokhov@students.uiuc.edu)} \\
        \vspace{3mm}
        \textit{University of Illinois at Urbana-Champaign} \\
        Francois Ostiguy (supervisor) \\
        \textit{Fermi National Accelerator Laboratory} }
\date{August 6, 1999}

\maketitle

\begin{abstract}
The report summarizes the work that was done while working at Fermilab
during the summer of 1999. The main task was the creation of a parser of
the command language of MAD program and a converter from that language to
C++ definitions in CFG format. In addition to a summary of what was done,
the report contains some relevant details about the parser and examples of
converted code. It also describes some of the problems that were encountered
and the solutions found for them.
\end{abstract}

\section{Introduction}
During the summer of 1999 I have worked as a summer student at the Beam
Physics Department at Fermi National Accelerator Laboratory under the
supervision of Dr.~Francois Ostiguy. The job involved working on single
extensive project. The project consisted of working on a program that
would parse MAD (Methodical Accelerator Design) definitions and commands,
store the defined objects in the memory while ignoring the commands, and
output the definitions in the CFG format (code to be included in the
\texttt{init} method of the Accelerator class), which uses classes from a C++
library called Beamline/Mxyzptlk. Significant advice and collaboration
during the work on the parser was provided by Dr.~Oleg Krivosheev
(kriol@fnal.gov).

\section{Parser Requirements}
Several tools were used for developing the MAD parser. The C-language part
of the parser code was written in and tested with GCC v.2.8.1 compiler.
However, the code is fairly portable and should work with any
ANSI-C-compatible compiler. The lexer was created using Flex v.2.5.4
scanner generator. It should be fairly compatible with AT\&T Lex but
requires minor code changes. Bison v.1.27 parser generator was used to
create the parser. Again, it should be quite compatible with AT\&T Yacc,
but some small changes in the code might also be required. The program
consists of many files which can be compiled and linked using a provided
makefile, which is written for GNU Make v.3.77. In addition, C data structures
for storing the information about MAD objects were created using the library
Glib v.1.2.3, found on the GIMP Toolkit site (http://www.gtk.org). All the
tools mentioned above were made by GNU (http://www.gnu.org).

\section{Design Issues}
\subsection{Reading MAD Structure Into Memory}
The lexer that is built by Flex from regular-expression-based rules
recognizes MAD keywords, identifiers, numbers, strings, and comments. It
returns corresponding tokens and semantic values to the parser. The parser
contains the grammar for MAD definitions and uses it to recognize those
definitions and store them into internal data structures.

\subsection{Internal Massage}
After MAD definitions are stored into internal data structures and before
the actual CFG output takes place, there are several actions that need to
taken: checking for variable loops, sorting, and dependence resolutions.
Before the output takes place, constants, variables, beam elements, and beam
lines are sorted by the line number on which they were defined in the MAD file.
Then dependences are also checked and resolved by re-arranging the order
of the corresponding objects. At the time of the writing, a circular
definition (that is, a loop) can be detected by running the dependency
check, which will not properly terminate in case of such a definition.

\subsection{CFG Output}
A CFG file is created for the output of the definitions. All the CFG
output is done into this file. It is performed in four steps. First,
all the constants are output, then the variables, then the beam elements,
and finally the beam lines. This is done both for easier usage of the
resultant CFG file and to better reflect the language structure of MAD,
even though the definitions might not be in such separate blocks in
the original MAD file. In addition, comments (changed from MAD to C++
style) are also printed in all sections.

\section{Parser Internals}
As was mentioned above, almost all of the data structures that are
created by the parser to store MAD definitions come from the Glib library.
Hash tables are kept for storing information about constants, variables,
beam elements, and beam lines. In each hash table, the key is a pointer
to the name of the object and the value is a pointer to the corresponding
structure. The tables below show all these structures.

\begin{center}
\begin{tabular}{||l||}\hline\hline
Constant             \\ \hline
name                 \\
string value         \\
algebraic expression \\
line number          \\ \hline\hline
\end{tabular}
\hspace{2em}
\begin{tabular}{||l||}\hline\hline
Variable             \\ \hline
name                 \\
algebraic expression \\
line number          \\ \hline\hline
\end{tabular}
\\
\vspace{2em}
\begin{tabular}{||l||}\hline\hline
Beam Element \\ \hline
name         \\
kind         \\
length       \\
parameters   \\
line number  \\ \hline\hline
\end{tabular}
\hspace{2em}
\begin{tabular}{||l||}\hline\hline
Beam Line         \\ \hline
name              \\
beam element list \\
counter           \\
line number       \\ \hline\hline
\end{tabular}
\end{center}

Trees (GNode pointers in Glib) are employed by the parser for storing
algebraic expressions used by constants, variables, and beam elements.
Doubly-linked lists (GList pointers in Glib) are used for storing
information about the elements of a beam line. Finally, arrays of pointers
(GPtrArray pointers in Glib) are used for comments.

\section{What In MAD Is Handled}
MAD constant definitions are parsed and output to the CFG file. Constants
can be assigned algebraic expressions as well as string values. Built-in
constants from MAD ($\pi$, etc.) are predefined. All variables with arbitrary
algebraic expressions as allowed by MAD syntax are also parsed and output.
All beam element definitions are parsed, except for matrix and lump elements,
which are not fully parsed, meaning that they are recognized, but not all their
parameters are stored into memory. (For more details about what is not handled,
see the next section.) Beam line definitions are parsed and output as well,
including beam line expressions: inversion, inclusion, and replication.
Finally, MAD comments are handled by associating a comment that is on the same
line as a statement with that statement and by associating full-line comments
with the statement right after it.

\section{What In MAD Is NOT Handled}
Most importantly, the parser is designed for handling definitions only. Hence,
MAD commands are not interpreted. The lexer understands them, but the parser
essentially throws them away by simply outputting a message to the log file.
Multiple input files are also not handled. However, the user can concatenate
them by hand and use the result as the input for the parser. Another important
issue is short versions of keywords (for example, HMON instead of HMONITOR).
To successfully parse a MAD file the short versions should be converted to
their canonical names. Finally, as was briefly mentioned above, matrix and
lump beam elements are not fully handled because they do not fit into beam
element scheme and do not have direct Beamline/Mxyzptlk equivalents.

\section{Conversion Problems}
There are three main problems with converting from MAD to CFG, which are
results of the absence of direct correspondence between their elements.
First, some elements (for example, MULTIPOLE and YROT) are correctly stored in
the memory but are printed as comments to the CFG file. Second, other elements
(for example, SOLENOID) are correctly stored in the memory but are printed as
instances of fictitious classes. Third, elements like ELSEPARATOR and the
collimators are replaced by drifts. Output for the most elements includes
comments that tell about the above problems and changes. The comments also
list the values of parameters that don't have equivalents.

\section{A Conversion Example}
The converter was tested using version 18 of the Recycler lattice (r18). The
input is a 56KB MAD file. The only preliminary work that needed to be done was
the conversion of shortened beam element names into standard ones. A 100KB
output file was produced, and it looks valid. Notice that all the identifiers
are in upper case. This is because C++ is case-sensitive, while MAD is not.
Thus all MAD identifiers have to be converted to the same case.

\subsection{Constants}
No new MAD constants were declared in r18. Therefore, the converter produced
only one line:

\begin{verbatim}

// *** Constants ***

const double PI = 3.14159265358979323846;

\end{verbatim}

\subsection{Variables}
This is how these three variables were defined in the r18 MAD file:

\begin{verbatim}

! Define cell lengths for arc,dispersion suppressor and
! straight section
!
larccell = 17.288177764 !17.288180764-0.000003
lsscell = 17.288639
ldiscell = 12.96643319

\end{verbatim}

\noindent and how they were converted into CFG output:

\begin{verbatim}

// Define cell lengths for arc,dispersion suppressor and
// straight section
//
// 17.288180764-0.000003
double LARCCELL = 17.288177764;
double LSSCELL = 17.288639;
double LDISCELL = 12.96643319;

\end{verbatim}

\subsection{Beam Elements}
A beam element from r18 in MAD:

\begin{verbatim}

! Simulate magnet moves by placing H and V correctors at the
! end of the CFM magnets.
!
HC: hkicker, l=0.0, kick = 0.0, type=khka

\end{verbatim}

\noindent and the corresponding CFG output:

\begin{verbatim}

// Simulate magnet moves by placing H and V correctors at the
// end of the CFM magnets.
//
hkick HC ( (char*)"HC:khka", 0.0, 0.0 );

\end{verbatim}

\subsection{Beam Lines}
R18 beam lines in MAD:

\begin{verbatim}

IJHB30DS: line=(-IJHB30US)
IJHBINSRT: line=(IJHB30US,IJHB30DS)

\end{verbatim}

\noindent and in the CFG output file:

\begin{verbatim}

beamline IJHB30DS ( "IJHB30DS" );
         IJHB30DS.append( -IJHB30US );
beamline IJHBINSRT ( "IJHBINSRT" );
         IJHBINSRT.append( IJHB30US );
         IJHBINSRT.append( IJHB30DS );

\end{verbatim}


\section{How To Run The Converter}
After compilation there is only one executable to run, which is called
\texttt{mad2cfg}. If executed with no arguments, it prints out a message
telling how to use it:

\begin{verbatim}
Usage: mad2cfg [options] mad_file [cfg_file]
\end{verbatim}

There is only one option now, namely \texttt{-l}, which, if selected, commands
the program to create a log file called \texttt{madparser.log}. The input
file always has to be specified, while the output file argument is optional.
If it is not specified, the CFG output will go to STDOUT.

\section{Code Availability}
The code for the parser is available via read-only anonymous CVS from
the server reboot.fnal.gov. In order to be able to connect, set your
CVSROOT environment variable to \\

:pserver:anoncvs@reboot.fnal.gov:/usr/local/cvsroot \\

\noindent Then execute ``cvs login'' with password ``guest''. After the
login is successfully performed, you can get your private copy of the source
files by executing ``cvs checkout madparser''. Then you should check
GNUmakefile to see how include parameters are set, since you might need to
change them. Finally, run ``make'' to compile and build the MAD parser.

\section{Conclusion}
The MAD parser is fully functional, but there are certainly many features
that can be added to it. Handling multiple included files and outputting
in a number of different formats are two such features. Adding them requires
a little thought but can be done without a problem. Also, one might consider
using the parser to create a C++ factory that will produce instances of the
classes of the Beamline library. Finally, this summer has allowed me to learn
a great deal about construction of a lexer, a parser, an analyzer, and a
translator, as well as to obtain good experience in writing in ANSI C and
utilizing data structures in it.

\vspace{30mm}
\noindent Distribution:

O.~Krivosheev

E.~McCrory

L.~Michelotti

N.~Mokhov

F.~Ostiguy

M.~Syphers

\end{document}
